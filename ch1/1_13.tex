\documentclass{ctexart}
\usepackage{amsmath}
\date{} %不输出日期
\title{}
\author{}
\begin{document}
\maketitle

%{\em 证明}

\noindent 据提示算得$\phi=\frac{1+\sqrt{5}}{2}$,$\gamma=\frac{1+\sqrt{5}}{2}$

\noindent 易证$\phi^2=\phi+1$,$\gamma^2=\gamma+1$

\noindent 分别算$Fib(0)$,$Fib(1)$,$Fib(2)$,可知其符合等式
\begin{equation}
Fib(n)=\frac{(\phi^n-\gamma^n)}{\sqrt{5}}\label{tx}
\end{equation}
据归纳法,设$Fib(k) (k>=2)$符合等式\ref{tx}
\begin{equation}
\begin{aligned}
Fib(k + 1)&=Fib(k)+Fib(k+1) \\
	&=\frac{(\phi^k-\gamma^k)}{\sqrt{5}} + \frac{(\phi^{k-1}-\gamma^{k-1})}{\sqrt{5}} \\
	&=\frac{1}{\sqrt{5}}\left(\phi^k+\phi^{k-1}-\gamma^k-\gamma^{k-1}\right) \\
	&=\frac{1}{\sqrt{5}}\left[\phi^{k-1}\left(\phi+1\right)-\gamma^{k-1}\left(\gamma+1\right)\right] \\
	&=\frac{1}{\sqrt{5}}\left(\phi^{k+1}-\gamma^{k+1}\right)
\end{aligned}
\end{equation}

\noindent 因此
\begin{equation}
\frac{1}{\sqrt{5}}\phi^n=Fib(n)+\frac{1}{\sqrt{5}}\gamma^n
\end{equation}

\noindent 易算得
\begin{equation}
-\frac{1}{2}<\frac{1}{\sqrt{5}}\gamma^n<\frac{1}{2}
\end{equation}

\noindent 故$Fib(n)$是距离$\frac{1}{\sqrt{5}}\phi^n$最近的整数,相差小于$0.5$

\end{document}
